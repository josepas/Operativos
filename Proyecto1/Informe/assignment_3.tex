%%%%%%%%%%%%%%%%%%%%%%%%%%%%%%%%%%%%%%%%%
% Short Sectioned Assignment
% LaTeX Template
% Version 1.0 (5/5/12)
%
% This template has been downloaded from:
% http://www.LaTeXTemplates.com
%
% Original author:
% Frits Wenneker (http://www.howtotex.com)
%
% License:
% CC BY-NC-SA 3.0 (http://creativecommons.org/licenses/by-nc-sa/3.0/)
%
%%%%%%%%%%%%%%%%%%%%%%%%%%%%%%%%%%%%%%%%%

%----------------------------------------------------------------------------------------
%	PACKAGES AND OTHER DOCUMENT CONFIGURATIONS
%----------------------------------------------------------------------------------------

\documentclass[paper=a4, fontsize=11pt]{scrartcl} % A4 paper and 11pt font size

\usepackage[T1]{fontenc} % Use 8-bit encoding that has 256 glyphs
\usepackage{fourier} % Use the Adobe Utopia font for the document - comment this line to return to the LaTeX default
\usepackage[english]{babel} % English language/hyphenation
\usepackage{amsmath,amsfonts,amsthm} % Math packages

\usepackage{lipsum} % Used for inserting dummy 'Lorem ipsum' text into the template

\usepackage{sectsty} % Allows customizing section commands
\allsectionsfont{\centering \normalfont\scshape} % Make all sections centered, the default font and small caps

\usepackage{fancyhdr} % Custom headers and footers
\pagestyle{fancyplain} % Makes all pages in the document conform to the custom headers and footers
\fancyhead{} % No page header - if you want one, create it in the same way as the footers below
\fancyfoot[L]{} % Empty left footer
\fancyfoot[C]{} % Empty center footer
\fancyfoot[R]{\thepage} % Page numbering for right footer
\renewcommand{\headrulewidth}{0pt} % Remove header underlines
\renewcommand{\footrulewidth}{0pt} % Remove footer underlines
\setlength{\headheight}{13.6pt} % Customize the height of the header

%\numberwithin{equation}{section}  Number equations within sections (i.e. 1.1, 1.2, 2.1, 2.2 instead of 1, 2, 3, 4)
%\numberwithin{figure}{section}  Number figures within sections (i.e. 1.1, 1.2, 2.1, 2.2 instead of 1, 2, 3, 4)
%\numberwithin{table}{section}  Number tables within sections (i.e. 1.1, 1.2, 2.1, 2.2 instead of 1, 2, 3, 4)

%\setlength\parindent{0pt}  Removes all indentation from paragraphs - comment this line for an assignment with lots of text

%----------------------------------------------------------------------------------------
%	TITLE SECTION
%----------------------------------------------------------------------------------------

\newcommand{\horrule}[1]{\rule{\linewidth}{#1}} % Create horizontal rule command with 1 argument of height

\title{	
\normalfont \normalsize 
\textsc{Universidad Simon Bolivar} \\ [25pt] % Your university, school and/or department name(s)
\horrule{0.5pt} \\[0.4cm] % Thin top horizontal rule
\huge Sistemas Operativos \\ % The assignment title
\horrule{2pt} \\[0.5cm] % Thick bottom horizontal rule
}

\author{
  Jose Pascarella\\
  \texttt{11-10743}
  \and
  Daniela Socas\\
  \texttt{11-10979}
}

\date{6 de Febrero del 2015} % Today's date or a custom date

\begin{document}

\maketitle % Print the title


\section{Encriptacion}

El proyecto consistio en crear soluciones para la encriptacion y desencriptacion de texto plano usando dos algoritmos simples, Cesar y Murcielago, explicados en el enunciado. Para esto se crearon tres soluciones, una secuencial y dos soluciones concurrentes impletadas con hilos y procesos, para la ultima, la comunicacion entre procesos es realizada con archivos temporales. Todo esto con el fin de poner en practica la programacion concurrente aprendida en el laboratorio y comparar la complejidad de codificacion asi como el tiempo de corrida entre las tres implementaciones que acontinuacion se explican.


\subsection{Implementacion Secuencial}\newline


\emph{cript <-c/-d> <archivoEntrada> <archivoSalida>}\newline

Para esta solucion se recorrio linealmente el archivo de entrada caracter a caracter haciendo las dos transformaciones sucesivamente e imprimiendo cada caracter resultante en el archivo de salida hasta encontrar el fin del archivo.

Tiempo promedio de corrida para pruebas realizadas con 2 284 880 caracteres: 342012 nanosegundos.

\newpage


\subsection{Implementacion con Procesos}\newline

\emph{cript\_p <-c/-d> <NumeroDeHijos> <archivoEntrada> <archivoSalida>}\newline

Como fue explicada en el enunciado esta solucion crea la cantidad de procesos por hijo que se le especifique como parametro el proceso principal crea sus hijos intermedios y estos a su vez crean los hijos hojas que aplican el algoritmo que corresponda, luego los hijos intermedios esperan a que sus hijos terminen, concatenan los resultados que fueron guardados en los archivos hojas, aplican el algoritmo que corresponda y guardan en otro archivo temporal, por ultimo el proceso principal espera por los hijos intermedios para concatenar la solucion, omitir los espacios si es necesario y guardar todo en el archivo salida.

Tiempo promedio de corrida para pruebas realizadas con 2 284 880 caracteres y 3 hijos por proceso: 579042 nanosegundos.\newline
Tiempo promedio de corrida para pruebas realizadas con 2 284 880 caracteres y 10 hijos por proceso: 605053 nanosegundos.\newline


\subsection{Implementacion con Hilos}\newline

\emph{cript\_t <-c/-d> <NumeroDeHijos> <archivoEntrada> <archivoSalida>}\newline

La implementacion con hilos carga todos los caracteres del archivo en un arreglo y luego la funcionalidad es muy parecida a la implementacion con procesos, solo que ahora todos los hijos estan trabajando en un mismo arreglo, dado que los hijos estan dentro del mismo proceso comparten este arreglo y los demas datos. Al final el hilo raiz guarda el arreglo en el archivo salida omitiendo los espacios si es necesario.

Tiempo promedio de corrida para pruebas realizadas con 2 284 880 caracteres y 3 hijos por hilo: 648258 nanosegundos.\newline
Tiempo promedio de corrida para pruebas realizadas con 2 284 880 caracteres y 10 hijos por hilo: 676716 nanosegundos.\newline

\newpage

A continuacion las especificaciones del procesador donde se corrieron las pruebas:

\begin{verbatim}

Architecture:          i686
CPU op-mode(s):        32-bit, 64-bit
Byte Order:            Little Endian
CPU(s):                2
On-line CPU(s) list:   0,1
Thread(s) per core:    1
Core(s) per socket:    2
Socket(s):             1
Vendor ID:             GenuineIntel
CPU family:            6
Model:                 15
Stepping:              2
CPU MHz:               1667.000
BogoMIPS:              3324.97
Virtualization:        VT-x
L1d cache:             32K
L1i cache:             32K
L2 cache:              2048K

\end{verbatim}

\end{document}